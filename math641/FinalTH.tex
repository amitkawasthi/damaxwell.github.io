\documentclass[minion]{homework}
\usepackage{cmacros}
\usepackage{graphicx}
\usepackage{color}
\usepackage[all,cmtip]{xy}

\doclabel{Math F641: Midterm (Take Home Edition)}
\docdate{Due: 5:00pm December 15, 2017}
\begin{document}
\begin{aproblems}

\hproblem Suppose that $(f_n)$ is a sequence of continuous 
functions from $\Reals$ to $\Reals$ that converges uniformly
on $\Rats$.  Show that it converges uniformly on $\Reals$.

\hproblem Carothers 10.20 

\hproblem Carothers 8.79 

\hproblem Let
$$
X_K = \left\{f\in C([0,1]): \text{$f$ is Lipschitz with constant $K$ and $\int_0^1|f| \le 1$}\right\}.
$$
Show that $X_K$ is compact in $C([0,1])$.  Is $X_K$ also compact in $L_1([0,1])$?

\hproblem Let $\{f_n\}$ be a sequence of measurable real-valued functions.
Let $E=\{x: \text{$(f_n(x))$ converges}\}$.  Show that $E$ is measurable.


\hproblem (Riemann integrable functions are continuous almost everywhere.)
\begin{subproblems}
  \item Let $(\psi_n)$ be an increasing sequence of step functions with $|\psi_n|\le M$ for some $M$.  Show that $\lim \psi_n$ is continuous almost everywhere.
  \item Show that Riemann integrable functions are continuous almost everywhere.
  Hint:  Find functions $g$ and $G$ with $g\le f \le G$ where $G=g$ almost
  everywhere and where $g$ and $G$ are continuous almost everywhere.
\end{subproblems}

\hproblem (The approximate with wild abandon problem.)

Suppose $f\in L^1[a,b]$ and $\int_a^b fg=0$ for every
polynomial $g$.  Show that $f=0$ almost everywhere.

\Hint: First show that $\int_I f = 0$ for every interval in $[a,b]$.
Then show that $\int_E f=0$ for every measurable set in $[a,b]$. You might
find Exercise 18.35 (the ``even more is true'' part) to be handy, as well.

\hproblem Compute $\lim_{n\ra\infty} \int_0^\infty \left(1+\frac{x}{n}\right)^{-n}\cos(x/n)\;dx$.

\hproblem Given $f\in L^1(\Reals)$, let $f_t(x)=f(x-t)$.  Show that the
map taking $t$ to $f_t$ is continuous as a map from $\Reals$ to $L^1(\Reals)$.

\hproblem Consider the series $\sum_{k=1}^\infty a_k \sin(k x)$ on
the domain $[0,2\pi]$.  Suppose that $\sum_{k=1}^\infty (a_k)^2$ converges.
Prove that the series converges in $L^2([0,2\pi])$.
Compare this result with the first problem of the midterm.

\hproblem A sequence $(f_n)$ is Cauchy in measure if for every
$\epsilon>0$ there is an index $N$ such that if $n,m\ge M$ then
$m( \{|f_n-f_m|>\epsilon\} ) < \epsilon$.

Show that if $(f_n)$ is Cauchy in measure and has a subsequence
that is convergent in measure, then the full sequence is Cauchy in measure.

\end{aproblems}
\newpage

{\bf  Rules and format:}
\begin{itemize}

\item You are welcome to discuss this exam with me (David Maxwell) to ask for hints and so forth.
\item You may not discuss the exam with anyone else until after the due date/time.
\item You are permitted to reference Carothers but no other text, nor
may you consult the internet.
\item Each problem is weighted equally.
\item  If you find a suspected typo, please contact me as soon as possible and I will
communicate it to the class if needed.
\item The due date/time is absolutely firm.
\item We will hold a hint session during finals week, TBA.
\end{itemize}

\end{document}