\documentclass[minion]{homework}
\newcommand{\Reals}{\mathbb{R}}
\doclabel{Math F615: Homework 5}
\docdate{Due: February 22, 2019}
\usepackage{graphicx}

\newcommand{\bfx}{\mathbf{x}}
\newcommand{\bfv}{\mathbf{v}}

\begin{document}

\begin{problems}

\problem Consider the matrix
\[
A = \begin{pmatrix} 
23 & -8 & 4\\
21 & -8 & 5\\
-126 & 42 & -19\end{pmatrix}.
\]
\begin{subproblems}
\item Show that $v_1=[-1,-2, 3]$,  $v_2=[1, 3 , 0]$ and 
$v_3=[0,1,2]$ are eigenvectors of $A$, and determine their
associated eigenvalues.
\item Compute the solution of
\[
u'=Au
\]
with initial condition $u(0)=v_3$.  Show, by plugging your solution
into the ODE, that your solution really is a solution.
\item Compute the solution of
\[
u'=Au
\]
with initial condition $u(0)=v_2+v_3$.  Show, by plugging your solution
into the ODE, that your solution really is a solution.
\item Determine the exact solution of
\[
u'=Au
\]
with initial condition $u(0) = [1,5,5]$.
\end{subproblems}

\problem Suppose you wish to apply the RK4 method to solve the ODE
of the previous problem.  What is the largest time step you can use before 
issues concerning absolute stability arise in your solution?

\problem
\begin{subproblems} 
\item Use your Newton solver from last week's homework to implement
the trapezoidal rule for solving systems of ODEs.  
\item Determine the exact solution to the problem
\begin{equation}
\begin{aligned}
u' &= 1\\
v' &= v - u^2
\end{aligned}
\end{equation}
with initial condition $u(0)=0$ and $v(0)=1$.
\item Test your solver against the previous exact solution
and confirm that it has the predicted order of accuracy.
\end{subproblems}

\problem Consider this one-step (Runge-Kutta) method, the 
implicit midpoint method,
\begin{equation}
\begin{aligned}
u_* &= u_n + h f(t_n+h/2,u_*) \\
u_{n+1} &=  u_n + k f(t_n+h/2,u_*)
\end{aligned}
\end{equation}
The first equation (stage) is Backward Euler to determine an approximation to the value at the midpoint in time and the second stage is the midpoint method using this value.
\begin{subproblems}
\item Determine the order of accuracy of this method.
\item Determine the stability region.
\item Is this method A-stable? 
\end{subproblems}
\end{problems}
\end{document}