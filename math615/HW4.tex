\documentclass[minion]{homework}
\newcommand{\Reals}{\mathbb{R}}
\doclabel{Math F615: Homework 4}
\docdate{Due: February 15, 2019}
\usepackage{graphicx}

\newcommand{\bfx}{\mathbf{x}}
\newcommand{\bfv}{\mathbf{v}}

\begin{document}

\begin{problems}

\problem 
\begin{subproblems}
\item
Make a graph of the boundary of the absolute stability
region for the Runge-Kutta RK4 method on page 24.

\item Apply the RK4 method to $u'=30u(1-u)$ with u(0) = 0.1 on the 
interval $0\le t\le 2$. Use 14, 17 and 25 steps. For each run
graph the numerical solution and 
the exact solution on the same plot. 

\item Explain the previous sequence of graphs in terms of the ODE and
the plot from part a.  Your answer should contain a quantitative explanation
for why the transiton occurs at the value of $h$ you observe.
\end{subproblems}

\problem The $\theta$-method for finite difference solutions of ODEs is
\[
u_{n+1} = u_n + h[\theta f_{n+1} + (1-\theta) f_n]
\]
where $\theta\in[0,1]$.
\begin{subproblems}
\item Compute the order of convergence of this method.  Hint:
it depends on $\theta$.
\item Numerically graph the boundary of the absolute stability region
for $\theta=1/4$.
\item Show that the method is $A$-stable if and only if $\theta\ge 1/2$.
\item Bonus: Analytically compute the boundary of the absolute
stability region for $\theta=1/4$
\end{subproblems}

\problem Newton's method can be used to solve
\[
f(x)=0
\]
where $x\in\Reals^n$ and $f(x)\in \Reals^n$.
Starting from an initial guess $x_k$,
\[
x_{k+1} = x_k - Df(x_k)^{-1} f(x_k).
\]
Here, $Df(x)$ is the Jacobian matrix
\[
Df_{ij} = \frac{\partial f_i}{\partial x_j}
\]
Implement Newton's method for systems.  Your
function should take as arguments $f$, $Df$
and $x_0$ (an initial guess).  It should terminate
whenever either 
\begin{itemize}
\item $|f(x)|_\infty$ is less than a specified tolerance
\item $|f(x)|_\infty$ is less than a sepcified fraction of $|f(x_0)|_\infty$
\end{itemize}

These tolerances should be specified with optional arguments 
as used in your language of choice.

Test your code against TBA.

\problem The energy for the heat equation $u_t=u_{xx}$ 
for $0\le x\le 1$ is
\[
E(t) = \frac{1}{2}\int_0^1 (u_x(x,t))^2\;dx.
\]
\begin{subproblems}
\item Assuming that at $x=0$ and at $x=1$ $u$ satisfies either
a homogeneous Dirichlet condition or a homogeneous Neumann condition,
show that
\[
\frac{d}{dt} E(t) \le 0.
\]
Hint: Take a time derivative, use the PDE, and integrate by parts.
\item Conclude that the only solution of $u_t=u_{xx}$ with $u=0$
at $t=0$, and at $x=0$ and $x=1$ is the zero solution.
\end{subproblems}

\problem The backwards heat equation reads
\[
u_t = -u_{xx},
\]
so all that differs is a sign on the right-hand side.  But this sign makes all the difference.

We will work with this equation for $0\le x \le 1$ and $0\le t\le 1$, and
with homogeneous Dirichlet boundary conditions, so $u=0$ at $x=0$ and $x=1$.
\begin{subproblems}
\item Show that 
\[
v(t) = \sin(k\pi x) e^{k^2\pi^2 t}
\]
is a solution of the PDE and the boundary conditions.
\item For each $\epsilon>0$, find a solution of the PDE and boundary conditions
that satisfies $|u(0,x)|<\epsilon$ at each $x$, but $|u(1,x)|\ge 1$ at some $x$.
\item Suppose you wish to find the solution $u$ of the backwards heat equation with initial condition $u_0$.  But you don't know $u_0$ exactly, you know $\hat u_0$,
and that $|u_0(x)-\hat u_0(x)|<10^{-47}$ at every $x$.  So you solve the
backwards heat equation for $\hat u$ instead.
Find an $L$ such that $|u(x,1)-\hat u(x,1)|<L$ for all $x$, or
explain why no such $L$ exists.
\end{subproblems}

\problem Implement the explict method for solving the heat equation with
right-hand side function
\[
u_t=u_{xx} + f
\]
on $0\le x \le 1$ and $0\le t\le T$.  You function should have
the following signature:

forcedheat(f,u0,N,M)

where 
\begin{itemize}
	\item $f(x,t)$ is a function and provedes the desired forcing term 
	\item $u0(x)$ is a function and provides the desired
initial condition. 
    \item $N+1$ is the number of interior spatial steps
    \item $M$ is the number of time steps
\end{itemize}
It should return $(x,t,u)$ where $x$ is an array of grid coordinates
that includes $0$ and $1$, $t$ is a vector of $t$ coordinates that includes
$0$ and $T$, and where $u$ is an $(N+2)\times(M+1)$ matrix where column
$j$ encodes the solution at time $t_j$.

Test your code as follows

\begin{itemize}  
\item Compute what $f$ is if the solution is $u(t,x)=\sin(t)x(1-x)$.  
\item Now, working on $0\le x\le 1$ and $0\le t\le 2\pi$ 
compute solutions with this forcing term and compare your solution
with the exact solution.  By working with various grid sizes,
confirm that your code has the expected order of convergence.
\end{itemize}


\end{problems}
\end{document}