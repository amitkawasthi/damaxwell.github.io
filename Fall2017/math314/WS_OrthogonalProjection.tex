\documentclass[minion]{homework}
\usepackage{cmacros}
\usepackage{graphicx}
\usepackage{color}
\usepackage[all,cmtip]{xy}
\def\vv{\mathbf{v}}
\def\vw{\mathbf{w}}
\def\vx{\mathbf{x}}
\def\va{\mathbf{a}}
\def\vb{\mathbf{b}}
\def\vzero{\mathbf{0}}

% \newcommand\tomeW{{\color{red}\textbf{(W) (Hand this one in to David.)}}}
\newcommand\W{{\color{red}\textbf{(W) (Hand this one in to David.)}}}
\newcommand\tome{{\color{red}\textbf{(Hand this one in to David.)}}}
\doclabel{Math F314: Projection onto a Subspace}
\docdate{October 30, 2017}

\begin{document}

\begin{aproblems}
\vskip 0.5cm

The goal of this worksheet is for you construct an orthogonal
projection onto a subspace.

Let
\[
\begin{aligned}
\va_1 &= (1,1,1)\\
\va_2 &= (0,1,0)\\
\end{aligned}
\]

Let $V=\text{span}(\va_1,\va_2)$. We showed in class that
the projection matrix is
\[
P = A(A^t A)^{-1}A^t
\]
for any matrix $A$ where the columns of $A$ are a basis for $V$.

\aproblem Explain why the columns of $A=[\va_1,\va_2]$ form a basis for $V$.

\aproblem Form the matrix $A^t A$.  Then compute its inverse.

\aproblem Compute the full projection matrix.

\aproblem Use your projection matrix to project $(5,2,1)$ onto $V$
\vskip 1cm
\hrule
The fact is, we hate making inverse matrices, like $(A^tA)^{-1}$.  
In the $2\times 2$
case, this is OK.  Otherwise, it is a ton of work, and is typically not necessary.  For projection, we use a three step process to project $\vb$
onto $V$.
\begin{enumerate}
\item Form $A^t A$.
\item Solve $A^t A \vx = A^t\vb$.  You'll use elimination. For a real problem
you'll use LU factorization.  For a real, real problem a computer will
use LU factorization for you.
\item The projection is $A \vx$.
\end{enumerate}

\aproblem Use elimination to solve $A^t A \vx=A^t(5,2,1)$.

\aproblem Compute $A\vx$ and ensure your solution matches the solution
you found earlier,.

\end{aproblems}
\end{document}
