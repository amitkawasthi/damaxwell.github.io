\documentclass[minion]{homework}
\usepackage{cmacros}
\usepackage{graphicx}
\usepackage{color}
\usepackage[all,cmtip]{xy}
\def\vv{\mathbf{v}}
\def\vw{\mathbf{w}}
\def\vx{\mathbf{x}}
\def\vy{\mathbf{y}}
\def\vz{\mathbf{z}}
\def\va{\mathbf{a}}
\def\vb{\mathbf{b}}
\def\vzero{\mathbf{0}}

% \newcommand\tomeW{{\color{red}\textbf{(W) (Hand this one in to David.)}}}
\newcommand\W{{\color{red}\textbf{(W) (Hand this one in to David.)}}}
\newcommand\tome{{\color{red}\textbf{(Hand this one in to David.)}}}
\doclabel{Math F314: Best Fit to a Line}
\docdate{November 1, 2017}

\begin{document}

\begin{aproblems}
\vskip 0.5cm

The goal of this worksheet is for you construct a line
of best fit to some data points.

The big picture is the following.  If the system
\[
A\vz = \vb
\]
does not have a solution, because $\vb$ does not lie in the
column space of $\vb$, you can solve instead the \textit{normal}
equations 
\[
A^t A\vz = A^t \vb.
\]
This system will always have a solution, and the solution will be the
point $\vz$ in the column space of $A$ such that $A\vz$ is as close
to $\vb$ as possible, in the sense that the length
\[
||A\vz - \vb||
\]
is minimized.

We want to fit a line to the following $(x,y)$ pairs.

\[
\begin{aligned}
(1,3/2), (2,3), (3,0), (4,2)
\end{aligned}
\]

Yes, there is a fraction.  Bummer.

\aproblem Make a sketch, by hand or using Matlab, to visualize the data set.

\aproblem Set up, longhand, equations to solve for $m$ and $b$ to find
a line $y=mx+b$ that passes through each of these data points.  

\aproblem The equation from the previous step can be written in the form
\[
A\vz = \vb
\]
where $\vz=(m,b)$.  What is the matrix $A$? What is the vector $\vb$? 
(I.e., concretely write down what these object are in terms of actual numbers) 

\aproblem Explain why, just glancing at $A$, that you do not expect there to be a solution.

\aproblem Find a basis for the left-null space of $A$ and use it to 
verify that 
\[
A\vz = \vb
\]
does not have a solution.

\aproblem Instead, we will find a best fit in the following sense.
Given a line $y=mx+b$, it generates four data points at our
four $x$-coordinates:
\[
\hat y_k = m x_k + b
\]
where $(x_1,x_2,x_3,x_4) =  (1,2,3,4)$.  Let 
$(\bar y_1, \bar y_2, \bar y_3, \bar y_4) = (3/2,3,0,2)$.
We want to minimize the error between $\bar\vy$ that comes
from our original data and $\hat \vy$ that comes from the line,
in the sense that we want to minimize
\[
E = ||\hat\vy - \bar\vy||. 
\]
Rewrite this quantity so that it involves the matrix $A$ and the 
unknown vector $\vz=(m,b)$.

\aproblem Sketch, by hand, the lines corresponding to the following 
choices of $(m,b)$: $(0,0)$, $(0,3)$, $(1,0)$ and $(0,2)$.  
Which of these four lines do you think has the smallest value of $E$?
Then compute $E$ for each of these cases.

\aproblem Set up a linear equation to solve for a best fit $(m,b)$.

\aproblem Now solve it and see if it gives a reasonable answer.

\aproblem Challenge! Go back to your answer to problem 5.  Each
basis vector gives you a condition that $\vb$ must statisfy in
order for there to be a solution of $A\vz =\vb$.  Explain, in
terms of geometry, slopes, rises, runs or similar what these two
conditions actually are.

\end{aproblems}
\end{document}
