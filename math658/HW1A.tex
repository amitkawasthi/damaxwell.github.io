\documentclass[minion]{homework}
\newcommand{\Reals}{\mathbb{R}}
\doclabel{Math F665: Homework 1A}
\docdate{Due: September 5, 2018}

\begin{document}

\begin{aproblems}

\hproblem SR 1.1

\hproblem SR 1.2 (i) - (ii)

\hproblem The general linear group $GL(\Reals,3)$ is the set of $3\times 3$ invertible matrics.
In this exercise, we show that $E(\Reals^2)$ can be seen as a subgroup of $GL(\Reals,3)$.

If
\begin{equation}
i(x,y) = \begin{pmatrix} c & \mp s  \\
s & \pm c \\
\end{pmatrix} \begin{pmatrix} x \\ y \end{pmatrix} +\begin{pmatrix} t_x \\ t_y \end{pmatrix}
\end{equation}
we define
\begin{equation}
M_i = \begin{pmatrix} c & \mp s & t_x \\
s & \pm c & t_y \\
0 &  0 & 1\end{pmatrix}.
\end{equation}
\begin{subproblems}
\item Suppose $x,y\in \Reals$. Show that 
\begin{equation}
M_i \begin{pmatrix} x \\ y \\ 1 \end{pmatrix}
\end{equation}
has the form
\begin{equation}
\begin{pmatrix} a \\ b \\ 1 \end{pmatrix}
\end{equation}
and that $i(x,y)=(a,b)$.
\item Show that if $i_1$ and $i_2$ belong to $E(\Reals^2)$ then
\begin{equation}
M_{i_2\circ i_1} = M_{i_2} M_{i_1}.
\end{equation}
Note that on the right-hand side of this equation we are multiplying matrics.
\item  Conclude that if $i\in E(\Reals^2)$, then
\begin{equation}
M_{i^{-1}} = (M_i)^{-1}.
\end{equation}
\end{subproblems}

\end{aproblems}
\end{document}	
