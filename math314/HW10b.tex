\documentclass[minion]{homework}
\doclabel{Math F314: Linear Algebra}
\docdate{Homework 10 Supplement}
\usepackage{cmacros}
\newcommand{\vs}{\mathbf{s}}
\newcommand{\vx}{\mathbf{x}}
\newcommand{\vB}{\mathbf{B}}
\newcommand{\vb}{\mathbf{b}}
\newcommand{\vr}{\mathbf{r}}
\newcommand{\vy}{\mathbf{y}}
\newcommand{\vu}{\mathbf{u}}
\newcommand{\vv}{\mathbf{v}}
\newcommand{\vw}{\mathbf{w}}
\newcommand{\vzero}{\mathbf{0}}
\begin{document}
\begin{aproblems}
\aproblem
Consider the matrix
\[
A=\begin{pmatrix} 2 & 4 & 5 \\
2 & 6 & 10\\
3 & 7 & 11 \\
0 & 8 & 12
\end{pmatrix}
\]
\begin{subproblems}
\item Find a collection of vectors $\vw_k$ such that $A\vx=\vb$
has a solution if and only if $\vw_k\cdot\vb=0$ for each $k$.
\item Determine if $A\vx = (5,4,6,4)$ has a solution.  You 
do not need to find the solution, if it exists.
\end{subproblems}

\aproblem Find a basis for the orthogonal complement
of the plane in $\Reals^4$ spanned by $(4,6,7,8)$ and $(5,10,11,12)$.

\aproblem For a square matrix, explain why its null space and its
left null space have the same dimension.  Then answer the following:
is it possible that a square matrix $A$ has an inverse but $A^T$ does not?

\aproblem Suppose $\vv$ is a non-zero 
vector in $\Reals^4$ and let $A=\vv\vv^T$.  The questions
below concern an arbitrary choice of $\vv$. Still, to get a picture
of what is going on, you might find it helpful to pick a random vector
$\vv$ in $\Reals^4$ and examine the properties of that particular $A=\vv\vv^T$.
Nevertheless, your answers need to work in general, not just for one choice
of $\vv$.
\begin{itemize}
	\item Compute the dimension of the column space of $A$.  You should explain
	your answer in terms of the column perspective of matrix multiplication. 
	\item Compute the dimension of the null space of $A$.
	\item Explain why the left null space equals the null space for this matrix.
	\item True or false: $A\vx = \vzero$ if and only if $\vx$ is perpendicular
	to $\vv$.
	\item Does the column space equal the row space for this matrix?
	\item Find a condition on $\vv$ that ensures $A^2=A$.
\end{itemize}

\aproblem 

\end{aproblems}

\end{document}
