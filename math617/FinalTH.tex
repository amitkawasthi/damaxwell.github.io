% \documentclass[minion]{homework}
% \usepackage{cmacros}
% \def\calB{\mathcal{B}}
% \DeclareMathOperator{\Ker}{{\rm Ker}}
% \DeclareMathOperator{\Image}{{\rm Image}}
% \def\Fbb{\mathbb{F}}
% \begin{document}
% \doclabel{Math 617: Homework 11}
% \docdate{Due: April 20, 2012}
% \begin{aproblems}
% 
\documentclass[minion]{homework}
\usepackage{cmacros}
\doclabel{Math F617: Take-home Final Exam}
\docdate{Due May 4, 2018}

\begin{document}

Please see the rules on the second page.

\begin{aproblems}
\hproblem  Suppose $X$ and $Y$ are Banach space and $\{T_n\}$
is a sequence of invertible operators in $B(X,Y)$
converging to some  operator $T$.  Suppose moreover, 
there is a constant $C>0$ such that $||T_n(x)||\ge C||x||$
for all $x\in X$ and $n\in\Nats$.  Show that $T$ has
a continuous inverse.

\hproblem Suppose $\{x_n\}$ is a sequence that converges weakly to $x$
in a Banach space $X$.
Let $T\in B(X,Y)$ be a compact operator.  Show $T(x_n) \to T(x)$.

\hproblem Suppose $T\in B(X)$ for some Banach space and
\[
S = \prod_{k=1}^n (T-\mu_i I)
\]
for certain complex numbers $\mu_i$.  Show that $S$ is 
invertible if and only if each $T-\mu_iI$ is.

\hproblem Let $f\in C[0,1]$ and define $T_f:C[0,1]\to C[0,1]$ by
$T_f(g) = fg$.  Compute $\sigma(T_f)$ and $\sigma_p(T_f)$.

\hproblem If $I$ is a closed, bounded interval in $\Reals$, we define
$H^1(I)$ to be the closure of the smooth functions on $I$ with respect
to the norm
\[
||u|| = ||u||_2 + ||u'||_2.
\]
\begin{subproblems}
\item Show that for smooth functions, the identity map
from $H^1(I)$ to $C[0,1]$ is continuous.  Hint: The Fundamental Theorem of Calculus will be handy.
\item The previous subproblem shows that elements of $H^1(I)$ can be
identified with continuous functions.  But elements of $H^1(I)$
need not be smooth.  Indeed, show that $f(x)=|x|$ belongs to $H^1([-1,1])$.
\end{subproblems}

\hproblem A set $\{\phi_j\}$ in a Hilbert space $X$ is called a Riesz basis
if there is a continuous linear isomorphism $T:X\ra X$ such that
$\{\phi_j\}$ is the image under $T$ of an orthonormal basis.

A set $\{\psi_j\}$ is a Schauder basis if it is linearly independent
and if $\overline{\mathrm{Span}\{\psi_j\}}=X$.

\begin{subproblems}
\item Give an example of a Schauder basis  for $\ell_2$ that is not a Riesz basis.
\item Suppose $\{\phi_j\}$ is a Riesz basis.  Show that $\sum_{j=1}^\infty c_j\phi_j$ converges if and only if $\{c_j\}\in\ell_2$.
\end{subproblems}

\newpage
{\bf  Rules and format:}
\begin{itemize}

\item You are welcome to discuss this exam with me (David Maxwell) to ask for hints and so forth.
\item  If you find a suspected typo, please contact me as soon as possible and I will
communicate it to the class if needed.
\item You may not discuss the exam with anyone else until after the due date/time.
\item You are permitted to reference our course text, but no other references.
\item Each problem is weighted equally.
\item The due date/time is absolutely firm.
\item We will schedule a hints session at a time TBA.
\end{itemize}

% \hproblem Let $\mathcal{M}$ be a closed subspace of $L^2([0,1])$ that
% is contained in $C[0,1]$.
% \begin{subproblems}
% 	\item Show that there exists $A>0$ such that $||f||_\infty\le A ||f||_2$
% 	      for every $f\in \mathcal{M}$. Hint: Closed graph theorem.
% 	\item Show that for each $x\in [0,1]$ there exists $g_x\in \mathcal{M}$ such 
% 	that $f(x)=\left<f,g_x\right>$ for all $f\in\mathcal{M}$ and $||g_x||_2\le A^2$ for all $x\in [0,1]$.
% 	\item Show that the dimension of $\mathcal{M}$ is at most $A^2$. Hint: Consider an
% 	      orthonormal sequence $\{f_j\}$ and note that $\sum_{j=1}^n|f_j(x)|^2\le A^2$ for
% 	every $x$.
% \end{subproblems}

% \hproblem Suppose that $K:X\ra X$ is a compact linear map
% between Hilbert spaces, and suppose that $I-K$ 
% has trivial kernel.
% \begin{subproblems}
% \item Show that $I-K$ is bounded below.
% \item Show that the image of $I-K$ is closed.
% \item Extra credit:  Show that $I-K$ is onto.  Hint:
% Consider the subspaces $A_n = (I-K)^n(X)$ and apply
% Reisz's Lemma. 
% \end{subproblems}

% \hproblem For $n\in\Nats$, let $f_n(x)= \sin(n\pi x)$.
% \begin{subproblems}
% \item Show that $f_n$ is an orthonormal basis for $L^2([0,1])$.
% Hint: Think about Sturm-Louiville problems.  Your proof should 
% be careful but short.
% \item Exhibit the function 1 in terms of this basis.
% Include with your solution a plot of the sum of the first
% 10 nonzero terms in the series.
% \end{subproblems}

% \aproblem Let $I=(a,b)$ be a bounded open interval in $\Reals$.
% Suppose $u\in H^1(I)$.
% \begin{subproblems}
% \item Show that if $x,y\in I$ and $x<y$, then
% \[
% u(y)-u(x) = \int_x^y u'(s)\; ds.
% \]
% Hint: Approximate $u'$ by functions $g_n\in C(I)$
% and let $v_n(z)=u(x)+\int_x^z g_n(s)\; ds$

% \item Recall that a function on $I$ is H\"older continuous
% with exponent $\alpha\in (0,1]$ if for all
% $x,y\in I$, there is a constant $K>0$ such that
% \[
% \frac{\abs{f(x)-f(y)}}{\abs{x-y}^\alpha} \le K.
% \]
% Show that $u$ is H\"older continuous with exponent
% $\alpha=1/2$.
% \item Conclude that $u$ admits a continuous extension to
% $[a,b]$.  Moreover, show that there is a constant $C$ independent
% of $u$ such that $|u(a)|\le C ||u||_{H^1}$ (i.e. that
% the map taking $u$ to $u(a)$ is continuous.)
% \end{subproblems}

\newpage
% \aproblem Suppose $p\in C^1[a,b]$ is positive.
% Show that given $f\in L^2[a,b]$,
% and $T>0$, there exists a function $u:[0,T]\ra L^2[a,b]$
% such that
% \begin{enumerate}
% \item $u(0)=f$.
% \item $u$ is continuous on $[0,T]$.
% \item $u$ is differentiable on $(0,T]$.
% \item $\frac{d}{dt} u=(p u')'$; the function on the left-hand side
% is in $L^2[a,b]$ and the derivatives on the right-hand side
% are with respect to $x\in[a,b]$ in the  sense of distributions.
% \item For all $t>0$, $u(t,a)=0=u(t,b)$.
% \end{enumerate}

% \hproblem A set $\{\phi_j\}$ in a Hilbert space $X$ is called a Riesz basis
% if there is a continuous linear isomorphism $T:X\ra X$ such that
% $\{\phi_j\}$ is the image under $T$ of an orthonormal basis.
% \begin{subproblems}
% \item Give an example of a Schauder basis for $\ell_2$ that is not a Riesz basis.
% \item Suppose $\{\phi_j\}$ is a Riesz basis.  Show that $\sum_{j=1}^\infty c_j\phi_j$ converges if and only if $\{c_j\}\in\ell_2$.
% \end{subproblems}

% \hproblem Let $\mathcal{M}$ be a closed subspace of $L^2([0,1])$ that
% is contained in $C[0,1]$.
% \begin{subproblems}
% 	\item Show that there exists $A>0$ such that $||f||_\infty\le A ||f||_2$
% 	      for every $f\in \mathcal{M}$. Hint: Closed graph theorem.
% 	\item Show that for each $x\in [0,1]$ there exists $g_x\in \mathcal{M}$ such 
% 	that $f(x)=\left<f,g_x\right>$ for all $f\in\mathcal{M}$ and $||g_x||_2\le A^2$ for all $x\in [0,1]$.
% 	\item Show that the dimension of $\mathcal{M}$ is at most $A^2$. Hint: Consider an
% 	      orthonormal sequence $\{f_j\}$ and note that $\sum_{j=1}^n|f_j(x)|^2\le A^2$ for
% 	every $x$.
% \end{subproblems}

\end{aproblems}

\end{document}
