\documentclass[minion]{homework}
\usepackage{cmacros}
\newcommand{\maple}[1]{{\tt\bf #1}}
\doclabel{Math F665: Homework 2}
\docdate{Due: September 19, 2018}

\begin{document}

\begin{aproblems}

\hproblem SR 4.3

\hproblem SR 4.4

\hproblem The interval between events $E_1=(t_1,x_1)$ and $E_2=(t_2,x_2)$ in some inertial coordinate
system is
\begin{equation}
c^2(t_1-t_2)^2 - (x_1-x_2)^2.
\end{equation}
Suppose $\iota:\Reals^2\ra\Reals^2$ is a transformation that preserves the interval between any two
events.  Assuming that $\iota$ is affine, show that there is a (possibly non-proper or non-orthochronus) Lorentz transformation $L$ and a vector $b\in \Reals^2$ such that
\begin{equation}
\iota(E) = {\mathcal C}^{-1} L {\mathcal C} E + b.
\end{equation}
Here $\mathcal C$ is the $2\times 2$ diagonal matrix with diagonal entries $c$ and $1$.
Hint: This problem should feel very familiar! And take advantage of problem 4.2!

\hproblem For simplicity the following problem is to be done in one space dimension.
Suppose in the frame of some inertial observer a function has the form
\begin{equation}
f(t,x) = \sin(\omega t)
\end{equation}
for some angular frequency $\omega$.  Now consider the frame of some observer traveling
with velocity $v$ relative to the original frame.  Determine the time difference between peaks
of the function as seen by the boosted observer.

\hproblem Pions are subatomic particles with a half life of $\Delta t=1.8\times10^{-8}$ seconds.  As a consequence, given a collection of pions left alone for a time $\Delta t$, half
of the pions will decay into other particles.

A beam of pions is traveling at a speed $v=0.99c$. Notice that in time $\Delta t$ the beam
travels
\begin{equation}
\Delta x = 0.99 c \Delta t = 5.35 \mathrm{m}
\end{equation}
and one might expect that the beam diminishes in intensity by one half every 5.35m.  Instead,
it deminishes by one half every 38m or so.  Explain the discrepancy.
\end{aproblems}

\end{document}