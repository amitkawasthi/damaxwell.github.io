\documentclass[minion]{homework}
\usepackage{cmacros,tensor}
\newcommand{\maple}[1]{{\tt\bf #1}}
\doclabel{Math F665: Homework 10}
\docdate{Due: November 12, 2018}
\def\ul#1{\underline{#1}}
\def\Cplx{\mathbb{C}}
\begin{document}


\begin{aproblems}

\hproblem Suppose that $\alpha^a(s)$ is a geodesic.  Show that
$\alpha^s(ks)$ is for any $k\in\Reals$.

\hproblem Suppose $X^a$ and $Y^a$ are parallel transported
along $\alpha^c(s)$.  Show that $g_{ab} X^a Y^b$ is constant.

\hproblem GR 5.6

\hproblem Use the result from GR 5.6 to show that the recipe
\begin{equation}
\nabla_a X^b = \frac{\partial X^b}{\partial x^a} + \Gamma^b_{ac} X^c
\end{equation}
is a tensorial recipe.

\hproblem  One way to think about Christoffel symbols is to
consider $\Gamma^a_{bc}$ as a collection of matrices $M\indices{^a_b}$,
one for each index $c$.  Show that the following algorithm
can be used to compute $\Gamma^a_{bc}$.
\begin{enumerate}
\item Let $u_b$ be the entries of row $c$ of $g_{ab}$.
\item Let $A_{ab} = \partial_a u_b$.
\item Let $B_{ab} = A_{ab} - A_{ba}$.
\item Let $C_{ab} = \partial_c g_{ab}$.
\item Let $D_{ab} = (1/2)( C_{ab}-B_{ab} )$.
\item Let $M\indices{^a_b} = g^{ac}D_{cb}$.
\end{enumerate}
Then $\Gamma^a_{bc} = M\indices{^a_b}$.

\end{aproblems}

\end{document}