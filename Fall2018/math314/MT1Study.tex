\documentclass[minion]{homework}
\usepackage{cmacros,graphicx}
\newcommand\Irr{\mathbb{I}}
\newcommand{\vs}{\mathbf{s}}
\newcommand{\vx}{\mathbf{x}}
\newcommand{\vB}{\mathbf{B}}
\newcommand{\vb}{\mathbf{b}}
\newcommand{\vr}{\mathbf{r}}
\newcommand{\vy}{\mathbf{y}}
\newcommand{\vu}{\mathbf{u}}
\newcommand{\vv}{\mathbf{v}}
\newcommand{\vzero}{\mathbf{0}}
\newcommand{\vw}{\mathbf{w}}

\DeclareMathOperator{\erf}{\rm erf}
\doclabel{Math F314: Midterm 1 Study Ideas}
\docdate{October 2, 2017}

\def\exercise{{\bf Exercise:}\par}

\begin{document}

Here are some things you should know for the midterm, which will cover all of chapters 1 and 2.  Everything from the homework and the quizzes is, in particular, fair game.  You must know all of that.  Here are more study ideas.

Given vectors $\vv_1$, $\vv_2$, and $\vv_3$, what is a linear combination of these vectors?

Describe all the linear combinations of $(1,1)$ and $(0,1)$.

Describe the set of all vectors $s(1,1)+t(-1,2)$ such that $s>0$ and $t<0$.

What is the definition of $\vv\cdot \vw$?

What is the definition of the cosine of the angle between $\vv$ and $\vw$?

When is $\vv$ perpendicular to $\vw$.

If $\vv$ and $\vw$ are unit vectors, what is the geometric interpretation of $\vv\cdot \vw$?

How is the length of a vector associated with the dot product?

What is the Cauchy-Schwartz inequality?

What is the triangle inequality?

Given two vectors in $\mathbb R^5$, how do you compute the angle between them?

If $\vv$ has length 1 and $\vw$ has length 2, what are the longest and shortest $\vv+\vw$ can be?

If $A=[\vv_1,\cdots,\vv_n]$ and $\vx=(x_1,\ldots,x_n)$, how is matrix
multiplication $A\vx$ defined?

Express the following problem using matrix multiplication:  Find a linear
combination of $(1,0,1)$, $(2,2,2)$, and $(7,0,7)$ that equals $(9,0,7)$.

Express the following problem using matrix multiplication:  Find the
intersection of the planes $z=1$, $x+y-2z=4$, and $2x+2y+1=0$.

Use elimination to convert a linear system to an equivalent upper-triangular linear system.

Use back-substitution to solve an upper-triangular linear system.

Given an augmented matrix $[A\vb]$, find a matrix $B$ such that
$B[A\vb]$ is an upper-triangular system.

Know elimination matrices and row-exchange (i.e. transposition) matrices like the back of your hand.  If $E$ is an elimination matrix, know how to quickly
multiply $EA$.  

What happens if you multiply a matrix by an elimination matrix on the {\bf right}?  I.e. what is $AE$?  If $P$ is a permutation matrix that interchanges rows 3
and 5, what is $AP$?

If $A$ is a matrix and $B=[\vv_1,\ldots,\vv_n]$, what is $AB$? 
Use this definition to compute
$$
\begin{bmatrix} 4 & 1\\ -1 & 4\end{bmatrix} \begin{bmatrix} 6 &0 \\ 1 & 4\end{bmatrix}
$$

Express the third column of
$$
AB=\begin{bmatrix} 4 & 1 & 6& -1\\ 2& -1 & 4& 4 \\ 1 & 6 & -1& 2 \\ 7 & 2 & -3 & 6\end{bmatrix} \begin{bmatrix} 2 & 0 & 1& -4\\ 4& 9 & 6& 3 \\ 1 & 6& -1& 2 \\  4& 9 & 3& 7\end{bmatrix} 
$$
as a linear combination of the columns of $A$.

When is it legal to multiply $A\vx$?  When is it legal to multiply $AB$?

What are the row, column, and row-column perspectives on matrix multiplication?

Know how to divide matrices into blocks to be able to do block multiplication.

Given
$$
A=\begin{bmatrix} 1 & 3 & -1\\ 2 & 0 &1 \\ -1 & -1 & 3 \\ 2 & 0 & 5\end{bmatrix}\qquad
B=\begin{bmatrix} 2 & 1 & 1\\ 9 & 0 &7 \\ 4 & -2 & 5\end{bmatrix}
$$
and
$$
AB=\begin{bmatrix} * & ? & ?\\ * & ? &? \\ * & * & *\\ * & * & *\end{bmatrix}
$$
compute the subblock marked with $?$ marks by dividing $A$ and $B$ into appropriately sized blocks, and doing block multiplication to compute just the
subblock.

If $R_1,\ldots,R_n$ are the rows of $A$, used block multiplication to compute $AB$.

Be able to show that the following are true:
\begin{itemize}
	\item If $A$ is invertible, the only solution of $A\vx=\vzero$ is $\vx=\vzero$.
	\item If there is a non-zero solution $\vx$ of $A\vx=\vzero$, then
	$A$ is not invertible.
	\item If $A$ is invertible, the equation $A\vx=\vb$ has a solution.
	\item If the equation $A\vx=\vb$ does not have a solution, then
	$A$ is not invertible.
	\item If $A$ is invertible, then the equation $A\vx=\vb$ can have
	at most one solution.
	\item If $\vx_1$ and $\vx_2$ are two different solutions of
	      $A\vx=\vb$, then $A$ is not invertible.
\end{itemize}

How is the inverse of a matrix defined?

If $A$ is a $7\times 7$ matrix and $\vw$ is the $4^{\rm th}$ column of $A^{-1}$, what equation does $\vw$ solve?  If $R$ is the second row of $A^{-1}$, what is $RA$?

Know how to compute a matrix inverse using Gauss-Jordan elimination.

Know how to factor $A=LU$.

Given a factorization $A=LU$, be able to solve $A\vx=\vb$.

What is the transpose of a matrix?

$\vy \cdot (A\vx)=?\cdot \vx$

Given the description of a permutation matrix, write it down.  I.e. write down the $4\times 4$ permutation matrix that takes row 1 to row 3, row 3 to row 4, and row 4 to row 2.

Given a permutation matrix, find its inverse.

\end{document}
